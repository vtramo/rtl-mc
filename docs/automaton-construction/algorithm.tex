\documentclass{article}
\usepackage{minted}
\usepackage{booktabs}
\usepackage[shortlabels]{enumitem}
\usepackage{amsmath}
\usepackage{algpseudocode}
\usepackage{graphicx}
\usepackage{amsfonts}
\usepackage{amssymb}
\usepackage{mathtools}
\usepackage{stmaryrd}
\usepackage{tikz}
\usepackage{textcomp}
\usepackage{letltxmacro}
\usepackage[ruled,vlined,linesnumbered]{algorithm2e}
\newcommand{\floor}[1]{\left\lfloor #1 \right\rfloor}
\newcommand{\ceil}[1]{\left\lceil #1 \right\rceil}
\renewcommand{\algorithmicrequire}{\textbf{Input:}}
\SetArgSty{textnormal}
\LinesNumbered
\DontPrintSemicolon

\begin{document}
    Un \textit{Transition-based Nondeterministic Finite Automaton} (TNFA) è una tupla $\mathcal{A}=(S,\Sigma,S_0,R,Acc)$ con accettazione transition-based.
    \vspace{0.25cm}\\
    Da un TNFA $\mathcal{A}$ può essere derivato (per i nostri scopi) un \textit{State-based Nondeterministic Finite Automaton} (SNBA) $\mathcal{A'}$ con accettazione state-based ed etichettatura sugli stati nel seguente modo:
    \begin{itemize}

        \item $\mathcal{A}' = (S',S'_0,R',L, Acc')$ dove:
        \begin{itemize}
            \item $S' = \{s_t\,|\,t\in R\}$,
            \item $S'_0 = \{s_t\,|\,t=(s,a,s')\,\text{ e }\,s \in S_0\}$,
            \item $R' \subseteq S' \times S'$ è definita dalla seguente regola:
            \[
                \frac{
                    t=(s,a,s')\,\land\, t'=(s',b,s'')
                }
                {
                    s_t \longrightarrow s'_t
                }
            \]
            \item $L(s_t) = a$, se $t=(s,a,s')$,
            \item $Acc' = \{s_t\,|\,t\in Acc\}\subseteq S'$.
        \end{itemize}
    \end{itemize}

    \LinesNumbered
    \begin{algorithm}
        \caption{\texttt{backwardSNBA}($\mathcal{A}$): conversione in un SNFA trasposto dato un TNFA $\mathcal{A}$ in input}\label{alg:backwardsnba}
        \KwData{$\mathcal{A}=(S,\Sigma,S_0,R,Acc)$: TNFA $\mathcal{A}$ in input}\\
        \KwData{$outTransitionStases \gets \emptyset$: mappa che associa a ogni stato $s\in S$ i suoi transition states $s_t \in S'$ in uscita}\\
        \KwData{$inTransitionStases \gets \emptyset$: mappa che associa a ogni stato $s\in S$ i suoi transition states $s'_{t'} \in S'$ in entrata}\\
        \KwOut{SNFA $\mathcal{A}' = (S',S'_0,R',L, Acc')$ trasposto}
        \vspace{0.30cm}\\
        $\mathcal{A}' \gets (S'=\emptyset,S'_0 =\emptyset, R'=\emptyset, L=\emptyset, Acc' =\emptyset)$\;
        \For{$s\in S$} {
            \For{$(s, t, s')\in R(s)$} {
                \For{$stateDenotation \in \texttt{extractStateDenotations}(\mathcal{A}, t.\texttt{cond})$} {\label{alg:extractStateDenotations}
                    $s_t = \texttt{newTransitionState}(stateDenotation)$\;
                    $S' = S'\,\cup\, {s_t}$\;

                    \vspace{0.20cm}\\

                    \If{$s \in S_0$} {
                        $S'_0 = S'_0\,\cup\,s_t$\;
                    }\Else{
                        $outTransitionStates[s].\texttt{pushBack}(s_t)$\;
                    }
                    $inTransitionStates[s'].\texttt{pushBack}(s_t)$\;

                    \vspace{0.20cm}\\
                    \If{$t \in Acc$} {
                        $Acc' = Acc'\,\cup\,s_t$
                    }

                    \vspace{0.20cm}\\

                    \For{$s'_{t'} \in outTransitionStates[s']$} {
                        \If{$s'_{t'}\in Acc'$} {
                            $R'=R'\,\cup\, \texttt{markItAsAccepting}(\{s'_{t'},s_{t}\})$\;\label{alg:markItAsAccepting1}
                        }\Else {
                            $R'=R'\,\cup\, (\{s'_{t'},s_{t}\})$
                        }
                    }
                }
            }

            \For{$s'_{t'} \in inTransitionStates[s]$} {
                \For{$s_{t} \in outTransitionStates[s]$} {
                    \If{$s_{t} \in Acc'$} {
                        $R'=R'\,\cup\, \texttt{markItAsAccepting}(\{s_{t},s'_{t'}\})$\;\label{alg:markItAsAccepting2}
                    }\Else{
                        $R'=R'\,\cup\, (\{s_{t},s'_{t'}\})$\;
                    }
                }
            }
        }
        \Return $\mathcal{A}'$
    \end{algorithm}

    \newpage

    L'Algoritmo~\ref{alg:backwardsnba} prende in input un TNFA $\mathcal{A}$ risultato della traduzione di una formula
    LTL in un automa e lo converte in un SNBA $\mathcal{A'}$ trasposto.
    L'implementazione esegue la conversione visitando l'automa $\mathcal{A}$ una sola volta.
    \vspace{0.15cm}\\
    Chiamiamo uno stato $s_t \in S'$ dell'automa SNBA $\mathcal{A'}$ \textit{transition state} in quanto viene creato a partire da una
    transizione appartenente all'automa $\mathcal{A}$.
    Siccome la libreria Spot associa agli archi dell'automa delle condizioni, l'Algoritmo~\ref{alg:backwardsnba} interpreta una transizione $(s, t, s') \in R$
    come un insieme di transizioni.
    Tale insieme di transizioni deve essere estrapolato da $t$. Da $t$ vengono estrapolati i mintermini associati alla sua
    guardia $t.\texttt{cond}$, e per ognuno di essi viene creato un nuovo transition state $s_t$.
    Questo lavoro viene fatto dalla funzione \texttt{extractStateDenotations} in Linea~\ref{alg:extractStateDenotations}.
    Nella implementazione C++ viene calcolato anche l'interpretazione dello stato.
    \vspace{0.15cm}\\
    L'algoritmo mantiene traccia di alcune informazioni:
    \begin{itemize}
        \item i transition states $s_t \in S'$ in uscita da uno stato $s\in S$ (cioè quelli creati a partire dagli archi uscenti di $s$). Tale informazione viene mantenuta in una mappa associativa \textit{outTransitionStates}.
        \item i transition states $s'_{t'} \in S'$ in entrata in uno stato $s\in S$ (cioè quelli creati a partire dagli archi entranti di $s$). Tale informazione viene mantenuta in una mappa associativa \textit{inTransitionStates}.
    \end{itemize}
    \vspace{0.15cm}\\
    Inoltre, la libreria Spot, indipendentemente dal fatto che l'automa usi una accettazione basata sugli stati o sulle transizioni, l'appartenenza
    di uno stato a un insieme di accettazione viene sempre memorizzato sulle transizioni (come un bit vector).
    La convenzione di Spot per implementare la state-base acceptance è quella di marcare ogni arco uscente da uno stato come accettante.
    Per questo motivo, a Linea~\ref{alg:markItAsAccepting1} e a Linea~\ref{alg:markItAsAccepting2}, se un transition state $s_t$ è accettante,
    tutte le sue transizioni in uscita devono essere marcate come accettanti.
    \vspace{0.15cm}\\
    Infine, si noti che a Linea~\ref{alg:markItAsAccepting1} e a Linea~\ref{alg:markItAsAccepting2} gli archi vengono creati in ordine inverso in quanto vogliamo il trasposto di $\mathcal{A'}$.
    \newpage
    Di seguito vengono riportati due esempi di output.
    Le formule subiscono le solite trasformazioni (discretizzazione e LTL).
    \begin{itemize}
        \item Formula: $\texttt{G}(i) \land\, t_0 \,\land\, \texttt{G}(t_1)\, \land\, \texttt{F}(p \,\land\, \texttt{F}\,q)$\\
        Total states: 52\\
        Total final states: 8\\
        Total transitions: 416
        \item Formula: $p_1 \,\land\, q_1 \,\land\, \texttt{X}(p_1) \,\land\, \texttt{X}(q_1) \,\land\, (v_1\, \texttt{U}\, (r_1\, \texttt{R}\, z_1)) \,\land\, \texttt{G}(x_1) \,\land\, \texttt{F}(u_1 \,\land\, \texttt{F}(p2 \,\land\, (\texttt{F} p_3\,\lor\,(u_2\,\texttt{W}\, p_4)))) \,\land\, (t\, \lor\, \texttt{G}(\texttt{X}(w)))$\\
        Total states: 69857\\
        Total final states: 0\\
        Total transitions: \color{red}{141222368}\\
    \end{itemize}
\end{document}